\documentclass{article}
\usepackage[top=20mm,left=20mm,bottom=20mm,right=20mm]{geometry}
\usepackage[version=3]{mhchem}
\usepackage{newtxtext,newtxmath}


\begin{document}
\section{Introduction}
\ce{Nb3Sn} is widely used for high field application such as ITER and NMR because of its high field performance. This is expected to play a vital role for next generation high field application; DEMO, FCC and so on. They demand very high $J_\mathrm{c}$ performance, so further $J_\mathrm{c}$ improvement is essential, but optimization of cross sectional layout and heat treatment regime is studied in detail, so \ce{Nb3Sn} superconductors seems to be attained fully $J_\mathrm{c}$ performance.

On the other hand, recent works evident grain morphology and Sn composition in \ce{Nb3Sn} layer contribute remarkably for $J_\mathrm{c}$ performance.From view point of crystallography, we have been challenge to improve $J_\mathrm{c}$ performance by doping additional element to Cu matrix, Nb core or Sn core on \ce{Nb3Sn} internal tin wire.

Zn is most attractive element in terms of \ce{Nb3Sn} growth kinetic. Zn seems to push out Sn into Nb, resulting in enhancing the growth rate of \ce{Nb3Sn} layer. More, Ti is promising element for $J_\mathrm{c}$ improvement because many researchers reported that Ti refine \ce{Nb3Sn} grain morphology; reduction of grain size and grain size scattering and promote Sn diffusion. In this work, we study about effect of Ti doping to Nb core on \ce{Nb3Sn} internal tin wire. We investigate \ce{Nb3Sn} grain morphology and diffusion reaction behavior in detail and measure $J_\mathrm{c}$-$B$ performance. Finally, we discuss about correlation between \ce{Nb3Sn} microstructure and superconducting performance.

\section{Experimental}
\subsection{Samples}
Fabrication process of the wire using in this study was based on so-called double-stacking process. The detail of procedure is written in []. In this study, Nb-Ti alloy which composition is 1.0 at\% and 1.54 at\% is used as Nb core. These wires are named as NT1-684, NT1.54-684 respectively.

We also prepare 1980 Nb-1.54at\% cores specimen and named as NT154-1980.12Zn-684, 12Zn-1980 wire are the reference wires using Cu-12wt\%Zn matrix.The first heat treatment is performed at 550$^\circ$C/100 h + 650$^\circ$C/100 h. The second is performed at 670$\sim$750$^\circ$C/50$\sim$200 h.

\subsection{Microstructural analysis}
\ce{Nb3Sn} grain morphology is observed with the fractured surfaces which is observed by field emission scanning electron microscope (FE-SEM). Element compositional analysis is performed by energy dispersive X-ray spectroscopy (EDX) on the polished cross section of the wires.
\end{document}
